Invariant features descriptions have been used and studies extensively in Computer Vision field. Histograms of colours are used in \cite{SVM_VC}, together with SVM to perform Image classification, with good accuracy results. Studies done in \cite{dalal2005histograms} prove the efficiency of HOG as invariant representation of images for pedestrian detection. Other popular methods of invariant representation like SIFT are described in \cite{DavidG}.  From this simple representations, more complex representations have been explored by combining or adding granularity layers, such as in [\cite{Spatial_Pyramid}, \cite{grauman2007pyramid}], where spatials pyramids of HOGS are used. More complex feature representations of images such as the Pyramid of HOGs imply that the feature representation is no longer a vector in R, and thus traditional kernels like linear or RBF kernels are not useful anymore. This leads to explore the use of alternative kernels such as the Pyramid Match Kernel [\cite{grauman2007pyramid}], that are able to deal with hierarchical representations of images with HOGs.
With respect to the use of SVM and kernelized methods for image recognition, there is also a lot of literature, mostly between 2000 where firsts good results were obtained for image classification [\cite{SVM_VC}] and 2010 or so, when Deep Learning proved to be more efficient for Computer visions. The most advanced methods using kernels for image processing are methods that use kernels which rather than taking a vector as input, are able to process hierarchical structures of data that are able to represent the inherent hierarchical nature of an image [\cite{Spatial_Pyramid},\cite{grauman2007pyramid}, \cite{bosch2007representing}]. 



